% Miguel A. Carreira-Perpinan's LaTeX macros.
% Time-stamp: <17/07/10 15:04:36 mcarreira-perpinan>

% Definitions that require only standard LaTeX2e things

% Test whether an argument is empty: []
\def \ifempty#1{\def\temp{#1} \ifx\temp\empty }


% Letters used for matrices and vectors (boldface), for functions (roman), etc.
\newcommand{\A}{\ensuremath{\mathbf{A}}}
\newcommand{\B}{\ensuremath{\mathbf{B}}}
\newcommand{\C}{\ensuremath{\mathbf{C}}}
\newcommand{\D}{\ensuremath{\mathbf{D}}}
\newcommand{\E}{\ensuremath{\mathbf{E}}}
\newcommand{\F}{\ensuremath{\mathbf{F}}}
\newcommand{\G}{\ensuremath{\mathbf{G}}}
\newcommand{\HH}{\ensuremath{\mathbf{H}}}
\newcommand{\I}{\ensuremath{\mathbf{I}}}
\newcommand{\J}{\ensuremath{\mathbf{J}}}
\newcommand{\K}{\ensuremath{\mathbf{K}}}
\newcommand{\LL}{\ensuremath{\mathbf{L}}}
\newcommand{\M}{\ensuremath{\mathbf{M}}}
\newcommand{\N}{\ensuremath{\mathbf{N}}}
\newcommand{\OO}{\ensuremath{\mathbf{O}}}
\newcommand{\PP}{\ensuremath{\mathbf{P}}}
\newcommand{\Q}{\ensuremath{\mathbf{Q}}}
%\newcommand{\R}{\ensuremath{\mathbf{R}}} % PROSPER defines \R
\newcommand{\RR}{\ensuremath{\mathbf{R}}}
\renewcommand{\SS}{\ensuremath{\mathbf{S}}}
\newcommand{\T}{\ensuremath{\mathbf{T}}}
\newcommand{\U}{\ensuremath{\mathbf{U}}}
\newcommand{\V}{\ensuremath{\mathbf{V}}}
\newcommand{\W}{\ensuremath{\mathbf{W}}}
\newcommand{\X}{\ensuremath{\mathbf{X}}}
\newcommand{\Y}{\ensuremath{\mathbf{Y}}}
\newcommand{\Z}{\ensuremath{\mathbf{Z}}}
\renewcommand{\aa}{\ensuremath{\mathbf{a}}}
\renewcommand{\b}{\ensuremath{\mathbf{b}}}
\renewcommand{\c}{\ensuremath{\mathbf{c}}}
\newcommand{\dd}{\ensuremath{\mathbf{d}}}
\newcommand{\e}{\ensuremath{\mathbf{e}}}
\newcommand{\f}{\ensuremath{\mathbf{f}}}
\newcommand{\g}{\ensuremath{\mathbf{g}}}
\newcommand{\h}{\ensuremath{\mathbf{h}}}
\newcommand{\bk}{\ensuremath{\mathbf{k}}}
\newcommand{\bl}{\ensuremath{\mathbf{l}}}
\newcommand{\m}{\ensuremath{\mathbf{m}}}
\newcommand{\n}{\ensuremath{\mathbf{n}}}
\newcommand{\p}{\ensuremath{\mathbf{p}}}
\newcommand{\q}{\ensuremath{\mathbf{q}}}
\newcommand{\rr}{\ensuremath{\mathbf{r}}}
\newcommand{\sss}{\ensuremath{\mathbf{s}}}  % TIPA defines \s and LaTeX \ss!
\renewcommand{\t}{\ensuremath{\mathbf{t}}}
\newcommand{\uu}{\ensuremath{\mathbf{u}}}
\newcommand{\vv}{\ensuremath{\mathbf{v}}}
%\renewcommand{\v}{\ensuremath{\mathbf{v}}}
\newcommand{\w}{\ensuremath{\mathbf{w}}}
\newcommand{\x}{\ensuremath{\mathbf{x}}}
\newcommand{\y}{\ensuremath{\mathbf{y}}}
\newcommand{\z}{\ensuremath{\mathbf{z}}}
\newcommand{\0}{\ensuremath{\mathbf{0}}}
\newcommand{\1}{\ensuremath{\mathbf{1}}}

% Bold symbols and greek letters
\newcommand{\balpha}{\ensuremath{\boldsymbol{\alpha}}}
\newcommand{\bbeta}{\ensuremath{\boldsymbol{\beta}}}
\newcommand{\bdelta}{\ensuremath{\boldsymbol{\delta}}}
\newcommand{\bepsilon}{\ensuremath{\boldsymbol{\epsilon}}}
\newcommand{\bgamma}{\ensuremath{\boldsymbol{\gamma}}}
\newcommand{\binfty}{\ensuremath{\boldsymbol{\infty}}}
\newcommand{\bkappa}{\ensuremath{\boldsymbol{\kappa}}}
\newcommand{\blambda}{\ensuremath{\boldsymbol{\lambda}}}
\newcommand{\bmu}{\ensuremath{\boldsymbol{\mu}}}
\newcommand{\bnu}{\ensuremath{\boldsymbol{\nu}}}
\newcommand{\bphi}{\ensuremath{\boldsymbol{\phi}}}
\newcommand{\bpi}{\ensuremath{\boldsymbol{\pi}}}
\newcommand{\bpsi}{\ensuremath{\boldsymbol{\psi}}}
\newcommand{\brho}{\ensuremath{\boldsymbol{\rho}}}
\newcommand{\bsigma}{\ensuremath{\boldsymbol{\sigma}}}
\newcommand{\btau}{\ensuremath{\boldsymbol{\tau}}}
\newcommand{\btheta}{\ensuremath{\boldsymbol{\theta}}}
\newcommand{\bxi}{\ensuremath{\boldsymbol{\xi}}}
\newcommand{\bzeta}{\ensuremath{\boldsymbol{\zeta}}}

\newcommand{\bDelta}{\ensuremath{\boldsymbol{\Delta}}}
\newcommand{\bGamma}{\ensuremath{\boldsymbol{\Gamma}}}
\newcommand{\bLambda}{\ensuremath{\boldsymbol{\Lambda}}}
\newcommand{\bPhi}{\ensuremath{\boldsymbol{\Phi}}}
\newcommand{\bPi}{\ensuremath{\boldsymbol{\Pi}}}
\newcommand{\bPsi}{\ensuremath{\boldsymbol{\Psi}}}
\newcommand{\bSigma}{\ensuremath{\boldsymbol{\Sigma}}}
\newcommand{\bTheta}{\ensuremath{\boldsymbol{\Theta}}}
\newcommand{\bXi}{\ensuremath{\boldsymbol{\Xi}}}
\newcommand{\bUpsilon}{\ensuremath{\boldsymbol{\Upsilon}}}

% Blackboard bold
\newcommand{\bbC}{\ensuremath{\mathbb{C}}}
\newcommand{\bbH}{\ensuremath{\mathbb{H}}}
\newcommand{\bbN}{\ensuremath{\mathbb{N}}}
\newcommand{\bbR}{\ensuremath{\mathbb{R}}}
\newcommand{\bbS}{\ensuremath{\mathbb{S}}}
\newcommand{\bbZ}{\ensuremath{\mathbb{Z}}}
\newcommand{\bbE}{\ensuremath{\mathbb{E}}}
\newcommand{\bbT}{\ensuremath{\mathbb{T}}}


% Calligraphic
\newcommand{\calA}{\ensuremath{\mathcal{A}}}
\newcommand{\calB}{\ensuremath{\mathcal{B}}}
\newcommand{\calC}{\ensuremath{\mathcal{C}}}
\newcommand{\calD}{\ensuremath{\mathcal{D}}}
\newcommand{\calE}{\ensuremath{\mathcal{E}}}
\newcommand{\calF}{\ensuremath{\mathcal{F}}}
\newcommand{\calG}{\ensuremath{\mathcal{G}}}
\newcommand{\calH}{\ensuremath{\mathcal{H}}}
\newcommand{\calI}{\ensuremath{\mathcal{I}}}
\newcommand{\calJ}{\ensuremath{\mathcal{J}}}
\newcommand{\calK}{\ensuremath{\mathcal{K}}}
\newcommand{\calL}{\ensuremath{\mathcal{L}}}
\newcommand{\calM}{\ensuremath{\mathcal{M}}}
\newcommand{\calN}{\ensuremath{\mathcal{N}}}
\newcommand{\calO}{\ensuremath{\mathcal{O}}}
\newcommand{\calP}{\ensuremath{\mathcal{P}}}
\newcommand{\calR}{\ensuremath{\mathcal{R}}}
\newcommand{\calS}{\ensuremath{\mathcal{S}}}
\newcommand{\calT}{\ensuremath{\mathcal{T}}}
\newcommand{\calU}{\ensuremath{\mathcal{U}}}
\newcommand{\calV}{\ensuremath{\mathcal{V}}}
\newcommand{\calW}{\ensuremath{\mathcal{W}}}
\newcommand{\calX}{\ensuremath{\mathcal{X}}}
\newcommand{\calY}{\ensuremath{\mathcal{Y}}}
\newcommand{\calZ}{\ensuremath{\mathcal{Z}}}

% Relational operators
% \bydef puts ``def'' over the equals sign and means ``is by definition
% equal to''. Another possibility is to use the \triangleq symbol.
\newcommand{\bydef}{\stackrel{\mathrm{\scriptscriptstyle def}}{=}}
% \simbydef puts ``def'' over the ~ sign and means ``is by definition
% distributed as''.
\newcommand{\simbydef}{\stackrel{\mathrm{\scriptscriptstyle def}}{\sim}}
\newcommand{\proptobydef}{\stackrel{\mathrm{\scriptscriptstyle def}}{\propto}}

% Other functions
\newcommand{\ceil}[1]{\lceil#1\rceil}
\newcommand{\floor}[1]{\lfloor#1\rfloor}
% These accept delimiters as arguments: \big \Big \bigg \Bigg.
% Default: \left ... \right.
\newcommand{\abs}[2][]{%
  \ifempty{#1} {\left\lvert#2\right\rvert} \else {#1\lvert#2#1\rvert} \fi}
\newcommand{\norm}[2][]{%
  \ifempty{#1} {\left\lVert#2\right\rVert} \else {#1\lVert#2#1\rVert} \fi}

% Left superscript
% (from http://www.maths.univ-rennes1.fr/~edix/sgahtml/typesetting_rules.html)
\newcommand{\leftexp}[2]{{\vphantom{#2}}^{#1}{#2}}

% Box of text (for pictures, tables, etc.), without frames. Arguments:
%   #1: separation between rows of text as a multiple of LaTeX's default
%       (optional, default 1).
%   #2: position (one of t, b, c).
%   #3: format (one of l, r, c or even p{2cm}).
%   #4: text, which can include line breaks (\\).
\newcommand{\caja}[4][1]{{%
    \renewcommand{\arraystretch}{#1}%
    \begin{tabular}[#2]{@{}#3@{}}%
      #4%
    \end{tabular}%
    }}

% The amsart class has its own \keywords command, in which case we
% don't replace it.
\providecommand{\keywords}[1]{\vspace*{0.5\baselineskip}\par\noindent\textbf{Keywords:} #1}

% The logos (AmSLaTeX, BibTeX, etc.) are defined in texnames.sty.

% List environments
%
% "simplelist": useful for compact lists. Use it instead of "itemize".
% Optional argument: the bullet (default \triangleright). Other nice
% bullets are available in the Zapf Dingbats font (e.g. \ding{166} in
% the pifont package).
%
\newenvironment{simplelist}[1][$\triangleright$]%
{%
\begin{list}{#1}{
\vspace{-\topsep}
\vspace{-\partopsep}
\setlength{\itemindent}{0cm}
\setlength{\rightmargin}{0cm}
\setlength{\listparindent}{0cm}
\settowidth{\labelwidth}{#1}
\setlength{\leftmargin}{\labelwidth}
\addtolength{\leftmargin}{\labelsep}
\setlength{\itemsep}{0cm}
%\setlength{\leftmargin}{0cm}
%\setlength{\labelwidth}{0cm}
}%
}%
{%
\end{list}
\vspace{-\topsep}
\vspace{-\partopsep}
}

%
% enumthm: an enumerated list intended to appear in theorems and not
% to be nested. Each item is numbered in Roman numerals in parentheses.
% A suitable way to refer to an item is:
% Theorem~\ref{thm:pithagoras}(\ref{en:triangle}) states...
\newenvironment{enumthm}%
{\begin{enumerate}%
\renewcommand{\theenumi}{\roman{enumi}}%
\renewcommand{\labelenumi}{(\theenumi)}}%
{\end{enumerate}}

% Commands for the hyperref package (they also require the url package).
% In all cases the typesetting style is tt by default but can be changed
% with the \urlstyle command, e.g. \urlstyle{tt}.
%
% \MACPhref[text]{link} typesets text (default = link) in tt and
% hyperlinks to link, e.g. \MACPhref{http://www.dcs.shef.ac.uk/~miguel}.
\newcommand{\MACPhref}[2][\DefaultOpt]{\def\DefaultOpt{#2}%
  \href{#2}{\url{#1}}}
%
% \MACPmailto[text]{email} typesets text (default = email) in tt and
% hyperlinks to email as mailto:email, e.g. \MACPmailto{miguel@dcs.shef.ac.uk}.
\newcommand{\MACPmailto}[2][\DefaultOpt]{\def\DefaultOpt{#2}%
  \href{mailto:#2}{\url{#1}}}

% % Hyphenation
% \hyphenation{elec-tro-pa-la-tog-ra-phy}

% % From the amsbook.cls file version 2.04
% \hyphenation{acad-e-my acad-e-mies af-ter-thought anom-aly anom-alies
% an-ti-deriv-a-tive an-tin-o-my an-tin-o-mies apoth-e-o-ses
% apoth-e-o-sis ap-pen-dix ar-che-typ-al as-sign-a-ble as-sist-ant-ship
% as-ymp-tot-ic asyn-chro-nous at-trib-uted at-trib-ut-able bank-rupt
% bank-rupt-cy bi-dif-fer-en-tial blue-print busier busiest
% cat-a-stroph-ic cat-a-stroph-i-cally con-gress cross-hatched data-base
% de-fin-i-tive de-riv-a-tive dis-trib-ute dri-ver dri-vers eco-nom-ics
% econ-o-mist elit-ist equi-vari-ant ex-quis-ite ex-tra-or-di-nary
% flow-chart for-mi-da-ble forth-right friv-o-lous ge-o-des-ic
% ge-o-det-ic geo-met-ric griev-ance griev-ous griev-ous-ly
% hexa-dec-i-mal ho-lo-no-my ho-mo-thetic ideals idio-syn-crasy
% in-fin-ite-ly in-fin-i-tes-i-mal ir-rev-o-ca-ble key-stroke
% lam-en-ta-ble light-weight mal-a-prop-ism man-u-script mar-gin-al
% meta-bol-ic me-tab-o-lism meta-lan-guage me-trop-o-lis
% met-ro-pol-i-tan mi-nut-est mol-e-cule mono-chrome mono-pole
% mo-nop-oly mono-spline mo-not-o-nous mul-ti-fac-eted mul-ti-plic-able
% non-euclid-ean non-iso-mor-phic non-smooth par-a-digm par-a-bol-ic
% pa-rab-o-loid pa-ram-e-trize para-mount pen-ta-gon phe-nom-e-non
% post-script pre-am-ble pro-ce-dur-al pro-hib-i-tive pro-hib-i-tive-ly
% pseu-do-dif-fer-en-tial pseu-do-fi-nite pseu-do-nym qua-drat-ic
% quad-ra-ture qua-si-smooth qua-si-sta-tion-ary qua-si-tri-an-gu-lar
% quin-tes-sence quin-tes-sen-tial re-arrange-ment rec-tan-gle
% ret-ri-bu-tion retro-fit retro-fit-ted right-eous right-eous-ness
% ro-bot ro-bot-ics sched-ul-ing se-mes-ter semi-def-i-nite
% semi-ho-mo-thet-ic set-up se-vere-ly side-step sov-er-eign spe-cious
% spher-oid spher-oid-al star-tling star-tling-ly sta-tis-tics
% sto-chas-tic straight-est strange-ness strat-a-gem strong-hold
% sum-ma-ble symp-to-matic syn-chro-nous topo-graph-i-cal tra-vers-a-ble
% tra-ver-sal tra-ver-sals treach-ery turn-around un-at-tached
% un-err-ing-ly white-space wide-spread wing-spread wretch-ed
% wretch-ed-ly Eng-lish Euler-ian Feb-ru-ary Gauss-ian
% Hamil-ton-ian Her-mit-ian Jan-u-ary Japan-ese Kor-te-weg
% Le-gendre Mar-kov-ian Noe-ther-ian No-vem-ber Rie-mann-ian Sep-tem-ber}

%%% Local Variables: 
%%% mode: latex
%%% TeX-master: t
%%% End: 
